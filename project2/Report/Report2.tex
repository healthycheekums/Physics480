
\documentclass[10pt,showpacs,preprintnumbers,footinbib,amsmath,amssymb,aps,prl,twocolumn,groupedaddress,superscriptaddress,showkeys]{revtex4-1}
\usepackage{graphicx}
\usepackage{dcolumn}
\usepackage{bm}
\usepackage[colorlinks=true,urlcolor=blue,citecolor=blue]{hyperref}
\usepackage{color}
\usepackage{amsmath}
\usepackage{algorithm}
\usepackage[noend]{algpseudocode}
\usepackage{hyperref}





\begin{document}
\title{Project 2}
\author{Parker Brue}
\affiliation{Department of Physics and Astronomy, Michigan State University, East Lansing, MI 48823}

\maketitle

\section{Theory}
	\subsection{Schroedinger's Equation for two electrons in a three-dimesnional harmonic oscillator well, noninteracting}	
The radial part of Schroedinger's equation for one electron reads

\begin{equation*}
	-\frac{\hbar^2}{2 m} \left ( \frac{1}{r^2} \frac{d}{dr} r^2
	\frac{d}{dr} - \frac{l (l + 1)}{r^2} \right )R(r) 
	+ V(r) R(r) = E R(r).
\end{equation*}
$V(r)$ is the harmonic oscillator potential $(1/2)kr^2$ with
$k=m\omega^2$ and $E$ is
the energy of the harmonic oscillator in three dimensions.
The oscillator frequency is $\omega$ and the energies are

\begin{equation*}
	E_{nl}=  \hbar \omega \left(2n+l+\frac{3}{2}\right),
\end{equation*}
with $n=0,1,2,\dots$ and $l=0,1,2,\dots$.

Due to the transformation to spherical coordinates
$r\in [0,\infty)$.  
The quantum number
$l$ is the orbital momentum of the electron.  
% 
Substituting $R(r) = (1/r) u(r)$:
% 

\begin{equation*}
	-\frac{\hbar^2}{2 m} \frac{d^2}{dr^2} u(r) 
	+ \left ( V(r) + \frac{l (l + 1)}{r^2}\frac{\hbar^2}{2 m}
	\right ) u(r)  = E u(r) .
\end{equation*}
% 
boundary conditions are $u(0)=0$ and $u(\infty)=0$.

We introduced a dimensionless variable $\rho = (1/\alpha) r$
where $\alpha$ is a constant length:
% 

\begin{equation*}
	-\frac{\hbar^2}{2 m \alpha^2} \frac{d^2}{d\rho^2} u(\rho) 
	+ \left ( V(\rho) + \frac{l (l + 1)}{\rho^2}
	\frac{\hbar^2}{2 m\alpha^2} \right ) u(\rho)  = E u(\rho) .
\end{equation*}
% 
In stipulations of the project, $l=0$.
We inserted $V(\rho) = (1/2) k \alpha^2\rho^2$:

\begin{equation*}
	-\frac{\hbar^2}{2 m \alpha^2} \frac{d^2}{d\rho^2} u(\rho) 
	+ \frac{k}{2} \alpha^2\rho^2u(\rho)  = E u(\rho) .
\end{equation*}
And multiplied by $2m\alpha^2/\hbar^2$ on both sides:

\begin{equation*}
	-\frac{d^2}{d\rho^2} u(\rho) 
	+ \frac{mk}{\hbar^2} \alpha^4\rho^2u(\rho)  = \frac{2m\alpha^2}{\hbar^2}E u(\rho) .
\end{equation*}
The constant $\alpha$ was fixed after some algebra
so that


\begin{equation*}
	\alpha = \left(\frac{\hbar^2}{mk}\right)^{1/4}.
\end{equation*}
We defined

\begin{equation*}
	\lambda = \frac{2m\alpha^2}{\hbar^2}E,
\end{equation*}
so we were able to rewrite Schroedinger's equation as

\begin{equation*}
	-\frac{d^2}{d\rho^2} u(\rho) + \rho^2u(\rho)  = \lambda u(\rho) .
\end{equation*}

And evaluate it by considering the canon form of second derivative of a function $u$
\begin{equation}
	u''=\frac{u(\rho+h) -2u(\rho) +u(\rho-h)}{h^2} +O(h^2),
	\label{eq:diffoperation}
\end{equation}
where $h$ is defined as our step length.
Minimum and maximum values for the variable $\rho$ are
$\rho_{\mathrm{min}}=0$  and $\rho_{\mathrm{max}}$
(In our case, we set $\rho_{\mathrm{max}}$ to 7.0).

With a specified number of mesh points, $N$, 
$h$ can be defined as, with $\rho_{\mathrm{min}}=\rho_0$  and $\rho_{\mathrm{max}}=\rho_N$,

\begin{equation*}
	h=\frac{\rho_N-\rho_0 }{N}.
\end{equation*}
The value of $\rho$ at a point $i$ is then 
\[
\rho_i= \rho_0 + ih \hspace{1cm} i=1,2,\dots , N.
\]
Rewriting the Schroedinger equation for $\rho_i$ as

\[
-\frac{u_{i+1} -2u_i +u_{i-1}}{h^2}+\rho_i^2u_i=-\frac{u_{i+1} -2u_i +u_{i-1} }{h^2}+V_iu_i  = \lambda u_i,
\]
$V_i=\rho_i^2$ is the harmonic oscillator potential.

The diagonal matrix element is:
\begin{equation*}
	d_i=\frac{2}{h^2}+V_i,
\end{equation*}
and the non-diagonal matrix element is:
\begin{equation*}
	e_i=-\frac{1}{h^2}.
\end{equation*}
All non-diagonal matrix elements are equal, a constant.
The Schroedinger equation is now:

\begin{equation*}
	d_iu_i+e_{i-1}u_{i-1}+e_{i+1}u_{i+1}  = \lambda u_i,
\end{equation*}
$u_i$ is unknown. we rewrote this so we could solve for the matrix eigenvalues.
\begin{equation}
	\begin{bmatrix}d_0 & e_0 & 0   & 0    & \dots  &0     & 0 \\
		e_1 & d_1 & e_1 & 0    & \dots  &0     &0 \\
		0   & e_2 & d_2 & e_2  &0       &\dots & 0\\
		\dots  & \dots & \dots & \dots  &\dots      &\dots & \dots\\
		0   & \dots & \dots & \dots  &\dots  e_{N-1}     &d_{N-1} & e_{N-1}\\
		0   & \dots & \dots & \dots  &\dots       &e_{N} & d_{N}
	\end{bmatrix}  \begin{bmatrix} u_{0} \\
	u_{1} \\
	\dots\\ \dots\\ \dots\\
	u_{N}
\end{bmatrix}=\lambda \begin{bmatrix} u_{0} \\
u_{1} \\
\dots\\ \dots\\ \dots\\
u_{N}
\end{bmatrix}.  
\label{eq:sematrix}
\end{equation}
The values of $u$ at the two endpoints are known through the boundary conditions, allowing us to skip the involved rows and columns. we specified the matrix with our values for $d_i$ and $e_i$
\begin{equation}
	\begin{bmatrix} \frac{2}{h^2}+V_1 & -\frac{1}{h^2} & 0        \\
		-\frac{1}{h^2} & \frac{2}{h^2}+V_2 & -\frac{1}{h^2}     \\
		0   & -\frac{1}{h^2} & \frac{2}{h^2}+V_{N-1} &     \\
	
	\end{bmatrix}
	\label{eq:matrixse} 
\end{equation}
This is the matrix we performed calculations on for the non interacting case

	\subsection{Schroedinger's Equation for two electrons in a three-dimesnional harmonic oscillator well, interacting}	
We took the case of two electrons in a harmonic oscillator well, but
introduced an interaction via the repulsive Coulomb interaction.
The single-electron equation:

\begin{equation*}
-\frac{\hbar^2}{2 m} \frac{d^2}{dr^2} u(r) 
+ \frac{1}{2}k r^2u(r)  = E^{(1)} u(r),
\end{equation*}
$E^{(1)}$ is the energy within one electron.
For two electrons with no repulsive Coulomb interaction, the
Schroedinger equation evovles to

\begin{equation*}
\left(  -\frac{\hbar^2}{2 m} \frac{d^2}{dr_1^2} -\frac{\hbar^2}{2 m} \frac{d^2}{dr_2^2}+ \frac{1}{2}k r_1^2+ \frac{1}{2}k r_2^2\right)u(r_1,r_2)  =
\end{equation*}
\begin{equation*}
E^{(2)} u(r_1,r_2) .
\end{equation*}

To evaluate this further, we introduced the relative coordinate $\mathbf{r} = \mathbf{r}_1-\mathbf{r}_2$
and the center-of-mass coordinate $\mathbf{R} = 1/2(\mathbf{r}_1+\mathbf{r}_2)$.
The radial Schroedinger equation evolved to:
\begin{equation*}
\left(  -\frac{\hbar^2}{m} \frac{d^2}{dr^2} -\frac{\hbar^2}{4 m} \frac{d^2}{dR^2}+ \frac{1}{4} k r^2+  kR^2\right)u(r,R)  = E^{(2)} u(r,R).
\end{equation*}

By using an ansatz, $u(r,R) = \psi(r)\phi(R)$, the equations for $r$ and $R$ were separated, giving the energy through the sum
of the relative energy $E_r$ and the center-of-mass energy $E_R$:

\begin{equation*}
E^{(2)}=E_r+E_R.
\end{equation*}

After taking care of this, we added then the repulsive Coulomb interaction between two electrons,

\begin{equation*}
V(r_1,r_2) = \frac{\beta e^2}{|\mathbf{r}_1-\mathbf{r}_2|}=\frac{\beta e^2}{r},
\end{equation*}
with $\beta e^2=1.44$ eVnm.

The $r$-dependent Schroedinger equation evolved to

\begin{equation*}
\left(  -\frac{\hbar^2}{m} \frac{d^2}{dr^2}+ \frac{1}{4}k r^2+\frac{\beta e^2}{r}\right)\psi(r)  = E_r \psi(r).
\end{equation*}

Through a similar process in the noninteracting case, we were able to introduce a dimensionless variable $\rho = r/\alpha$, and reduce the Schroedinger's equation to:

\begin{equation*}
-\frac{d^2}{d\rho^2} \psi(\rho) 
+ \frac{1}{4}\frac{mk}{\hbar^2} \alpha^4\rho^2\psi(\rho)+\frac{m\alpha \beta e^2}{\rho\hbar^2}\psi(\rho)  = 
\frac{m\alpha^2}{\hbar^2}E_r \psi(\rho) .
\end{equation*}
Going further, We defined a new 'frequency'

\begin{equation*}
\omega_r^2=\frac{1}{4}\frac{mk}{\hbar^2} \alpha^4,
\end{equation*}
and fixed the constant $\alpha$ to
\begin{equation*}
\alpha = \frac{\hbar^2}{m\beta e^2},
\end{equation*}
resulting in

\begin{equation*}
\lambda = \frac{m\alpha^2}{\hbar^2}E.
\end{equation*}
This further reduced Schroedinger's equation to

\begin{equation*}
-\frac{d^2}{d\rho^2} \psi(\rho) + \omega_r^2\rho^2\psi(\rho) +\frac{1}{\rho} = \lambda \psi(\rho).
\end{equation*}
In our case, we treated $\omega_r$ as a parameter reflecting the strength of the oscillator potential, and used the values
$\omega_r = 0.01$, $\omega_r = 0.5$, $\omega_r =1$, and $\omega_r = 5$ in our evaluations.

        \subsection{Implementing the Jacobi Rotation Algorithm}
Before we implemented the Jacobi algorithm, first we defined some basic nomenclature. We defined the quantities $\tan\theta = t= s/c$, with $s=\sin\theta$ and $c=\cos\theta$ and 
\begin{equation*}\cot 2\theta=\tau = \frac{a_{ll}-a_{kk}}{2a_{kl}}.
\end{equation*}
Through manipulation of the angle, $\theta$, we got non-diagonal matrix elements of the transformed matrix 
$a_{kl}$ to become non-zero and
consequently, the quadratic equation (using $\cot 2\theta=1/2(\cot \theta-\tan\theta)$

\begin{equation*}
t^2+2\tau t-1= 0,
\end{equation*}
resulting in

\begin{equation*}
t = -\tau \pm \sqrt{1+\tau^2},
\end{equation*}
and $c$ and $s$ are easily obtained via

\begin{equation*}
c = \frac{1}{\sqrt{1+t^2}},
\end{equation*}
and $s=tc$.  

We started with an (n x n) orthogonal transformation matrix, with the stipulation that  $\bold{S^T}=\bold{S^{-1}}$.
Nonzero matrix elements are:

$s_{kk}=s_{ll}=c,s_{kl}=-s_{lk}=-s,s_{ii}=-s_{ii}= 1,  i\not{=}k, l$
\end{document}